%-----------------------------------------------------------------------------------------------------------------------------------------------%
%	The MIT License (MIT)
%
%	Copyright (c) 2021 Jitin Nair
%
%	Permission is hereby granted, free of charge, to any person obtaining a copy
%	of this software and associated documentation files (the "Software"), to deal
%	in the Software without restriction, including without limitation the rights
%	to use, copy, modify, merge, publish, distribute, sublicense, and/or sell
%	copies of the Software, and to permit persons to whom the Software is
%	furnished to do so, subject to the following conditions:
%	
%	THE SOFTWARE IS PROVIDED "AS IS", WITHOUT WARRANTY OF ANY KIND, EXPRESS OR
%	IMPLIED, INCLUDING BUT NOT LIMITED TO THE WARRANTIES OF MERCHANTABILITY,
%	FITNESS FOR A PARTICULAR PURPOSE AND NONINFRINGEMENT. IN NO EVENT SHALL THE
%	AUTHORS OR COPYRIGHT HOLDERS BE LIABLE FOR ANY CLAIM, DAMAGES OR OTHER
%	LIABILITY, WHETHER IN AN ACTION OF CONTRACT, TORT OR OTHERWISE, ARISING FROM,
%	OUT OF OR IN CONNECTION WITH THE SOFTWARE OR THE USE OR OTHER DEALINGS IN
%	THE SOFTWARE.
%	
%
%-----------------------------------------------------------------------------------------------------------------------------------------------%

%----------------------------------------------------------------------------------------
%	DOCUMENT DEFINITION
%----------------------------------------------------------------------------------------

% article class because we want to fully customize the page and not use a cv template
\documentclass[a4paper,11pt]{article}

%----------------------------------------------------------------------------------------
%	FONT
%----------------------------------------------------------------------------------------

% % fontspec allows you to use TTF/OTF fonts directly
% \usepackage{fontspec}
% \defaultfontfeatures{Ligatures=TeX}

% % modified for ShareLaTeX use
% \setmainfont[
% SmallCapsFont = Fontin-SmallCaps.otf,
% BoldFont = Fontin-Bold.otf,
% ItalicFont = Fontin-Italic.otf
% ]
% {Fontin.otf}

%----------------------------------------------------------------------------------------
%	PACKAGES
%----------------------------------------------------------------------------------------
\usepackage{url}
\usepackage{parskip} 	

%other packages for formatting
\RequirePackage{color}
\RequirePackage{graphicx}
\usepackage[usenames,dvipsnames]{xcolor}
\usepackage[scale=0.9]{geometry}

%tabularx environment
\usepackage{tabularx}

%for lists within experience section
\usepackage{enumitem}

% centered version of 'X' col. type
\newcolumntype{C}{>{\centering\arraybackslash}X} 

%to prevent spillover of tabular into next pages
\usepackage{supertabular}
\usepackage{tabularx}
\newlength{\fullcollw}
\setlength{\fullcollw}{0.47\textwidth}

%custom \section
\usepackage{titlesec}				
\usepackage{multicol}
\usepackage{multirow}

%CV Sections inspired by: 
%http://stefano.italians.nl/archives/26
\titleformat{\section}{\color{black}\Large\scshape\raggedright}{}{0em}{}[{\titlerule
% [.5pt]
}]
\titlespacing{\section}{0pt}{10pt}{10pt}

%for publications
\usepackage[backend=bibtex,style=authoryear,sorting=ynt,maxbibnames=5]{biblatex}

%Setup hyperref package, and colours for links
\usepackage[unicode, draft=false]{hyperref}
\definecolor{linkcolour}{rgb}{0,0.2,0.6}
\hypersetup{colorlinks,breaklinks,urlcolor=linkcolour,linkcolor=linkcolour}
\newcommand{\colhref}[3]{\href{#2}{\color{#1}{#3}}} % colored href
\addbibresource{citations.bib}
\setlength\bibitemsep{1em}

%for social icons
\usepackage{fontawesome5}

%debug page outer frames
%\usepackage{showframe}


% job listing environments
\newenvironment{jobshort}[2]
    {
    \begin{tabularx}{\linewidth}{@{}l X r@{}}
    \textbf{#1} & \hfill &  #2 \\[3.75pt]
    \end{tabularx}
    }
    {
    }

    \newenvironment{jobcustom}[3]
    {
    \begin{tabularx}{\linewidth}{@{}l X r@{}}
    \textbf{#1} & \hfill &  #2 \\[-2.5pt]
    % & &
    \textcolor{black!55!white}{\small #3} \\[3.75pt]
    \end{tabularx}
    }
    {
    }

\newenvironment{joblong}[2]
    {
    \begin{tabularx}{\linewidth}{@{}l X r@{}}
    \textbf{#1} & \hfill &  #2 \\[3.75pt]
    \end{tabularx}
    \begin{minipage}[t]{\linewidth}
    \begin{itemize}[nosep,after=\strut, leftmargin=1em, itemsep=3pt,label=--]
    }
    {
    \end{itemize}
    \end{minipage}    
    }

    \newenvironment{jobcustomlong}[4]
    {
    \begin{tabularx}{\linewidth}{@{}l X r@{}}
    \textbf{#1} \textit{#2} & \hfill &  #3 \\[-2.5pt]
    \textcolor{black!55!white}{\small #4} \\[2.5pt]
    \end{tabularx}
    \begin{minipage}[t]{\linewidth}
    \begin{itemize}[nosep,after=\strut, leftmargin=1.75em, itemsep=1pt,label={\small$\bullet$}]
    }
    {
    \end{itemize} \vspace{.325em}
    \end{minipage}   
    }

\newcommand{\calpoly}{\textcolor{black!55!white}{Cal Poly, San Luis Obispo}}
\usepackage{amssymb}
%----------------------------------------------------------------------------------------
%	BEGIN DOCUMENT
%----------------------------------------------------------------------------------------
\begin{document}

% non-numbered pages
\pagestyle{empty} 

%----------------------------------------------------------------------------------------
%	TITLE
%----------------------------------------------------------------------------------------

% \begin{tabularx}{\linewidth}{ @{}X X@{} }
% \huge{Your Name}\vspace{2pt} & \hfill \emoji{incoming-envelope} email@email.com \\
% \raisebox{-0.05\height}\faGithub\ username \ | \
% \raisebox{-0.00\height}\faLinkedin\ username \ | \ \raisebox{-0.05\height}\faGlobe \ mysite.com  & \hfill \emoji{calling} number
% \end{tabularx}

\begin{tabularx}{\linewidth}{@{} C @{}}
\Huge{Max Varverakis} \\[7.5pt]
\href{https://github.com/MaxVarverakis}{\raisebox{-0.05\height}\faGithub\ GitHub: MaxVarverakis} \ $|$ \ 
\href{https://www.linkedin.com/in/maxvarverakis/}{\raisebox{-0.05\height}\faLinkedin\ LinkedIn: MaxVarverakis} \ $|$ \ 
% \href{https://mysite.com}{\raisebox{-0.05\height}\faGlobe \ mysite.com} \ $|$ \ 
\href{mailto:mvarvera@calpoly.edu}{\raisebox{-0.05\height}\faEnvelope \ mvarvera@calpoly.edu} \ $|$ \ 
\href{tel:+12489433648}{\raisebox{-0.05\height}\faMobile \ +1 (248) 943 3648} \\
% \hspace{1.23in}\href{mailto:varvmax@gmail.com}{\raisebox{-0.05\height}\faEnvelope \ varvmax@gmail.com} % \ $|$ \ 
\end{tabularx}
%----------------------------------------------------------------------------------------
% EXPERIENCE SECTIONS
%----------------------------------------------------------------------------------------

%Interests/ Keywords/ Summary
% \section{Summary}
% This CV is automatically generated and deployed using the \href{https://github.com/jitinnair1/autoCV}{autoCV} template along with GitHub Actions such that a new version of the CV is compiled, published and ready for use when the cv.tex file is updated. For details, \href{https://github.com/jitinnair1/autoCV}{click here}.

%----------------------------------------------------------------------------------------
%	EDUCATION
%----------------------------------------------------------------------------------------
\section{Education}
\begin{tabularx}{\linewidth}{@{}l X@{}}
2022 -- 2024 & \textbf{M.S., Applied Mathematics} \hfill\calpoly\ (GPA: 3.88/4.0) \\

2019 -- 2024 & \textbf{B.S., Mathematics} \hfill\calpoly\ (GPA: 3.48/4.0) \\

2019 -- 2024 & \textbf{B.S., Physics} \hfill\calpoly\ (GPA: 3.48/4.0) \\
\end{tabularx}

%----------------------------------------------------------------------------------------
%	RESEARCH
%----------------------------------------------------------------------------------------
\section{Research Experience} % Don't put a space between each environment!

\begin{jobcustomlong}{Accelerator Physics Intern}{-- Computational Accelerator Physics}{January 2025 -- March 2025}{{SLAC} National Accelerator Laboratory $\vert$ FACET-II, SLAC Group}
    \item Performed jitter simulations of FACET-II lattice parameters using Impact-T and Bmad.
    \item Large parallelized simulations were carried out on the high performance computer clusters NERSC and S3DF.
    \item Performed convergence scan of Impact-T to determine optimal simulation resolution.
    \item Consulted with experts in the FACET-II cohort to understand the implications of jitter on beam quality.
    \item Learned essential accelerator physics concepts and technologies, such as beam optics and lattice design.
    \item Presented simulations results bi-weekly to the FACET-II start-to-end simulation group.
\end{jobcustomlong}
\begin{jobcustomlong}{Accelerator Physics Intern}{-- Theoretical and Computational Particle Physics}{Sept 2024 -- December 2024}{{SLAC} National Accelerator Laboratory $\vert$ Michael Peskin, Ph.D., SLAC Professor}
    % \item Studying the infinite momentum frame in QED to understand forward electrons in the $2\gamma\to e^+e^-$ process.
    \item Derived Feynman rules for QED in the light-front gauge using the Lagrangian density.
    \item Calculated helicity differential cross sections for $2\gamma\to e^+e^-$ and Compton scattering using Mathematica.
    \item Obtained Compton scattering amplitude and differential cross section using standard QFT techniques.
    \item Ran HiPACE++ simulations of positron plasma wakefield acceleration on {LUMI} to probe beam stability.
    \item Performed G4Beamline simulations of a liquid xenon positron target to better understand the energy deposition density profile throughout the volume. Target windows were also examined.
\end{jobcustomlong}
\begin{jobcustomlong}{Applied Mathematics Graduate Researcher}{-- Representation theory}{Sep 2023 -- Jun 2024}{\calpoly\ $\vert$ Sean Gasiorek, Ph.D., Assistant Professor}
    % \item Investigated representation theory in the context of mathematical/theoretical physics, employing [specific methodologies or techniques] to analyze [specific aspect/problem], resulting in [quantifiable outcome or insight achieved].
    \item Investigated representation theory in the context of mathematical/theoretical physics by studying literature in diverse areas, ranging from group and knot theory to quantum mechanics.
    % \item Examined the physical consequences of Lie groups, such as $\mathfrak{so}(3)$ in quantum mechanics, utilizing irreducible representations and Lie group generators to elucidate [specific phenomena or behavior], contributing to [quantifiable advancement or understanding] in the field.
    \item Utilized irreducible representations and Lie group generators to observe the emergence of discretization and conservation laws in quantum systems.
    % \item Explored the braid group and its representations in the context of anyons, culminating in a master's thesis that delved into [brief overview of the research problem or question], employing [specific methodologies or approaches] to [specific objectives or goals], ultimately contributing to [broader understanding, advancement, or application] in the field of [mathematical/theoretical physics].
    \item Explored the braid group and its representations in the context of anyons, culminating in a \colhref{blue!65!black}{https://digitalcommons.calpoly.edu/theses/2844/}{master's thesis} that covers a wide breadth of applications of representation theory in physics.
\end{jobcustomlong}
\begin{jobcustomlong}{Computational Accelerator Physics Intern}{-- Plasma wakefield acceleration}{Jun 2023 -- Nov 2023}{{SLAC} National Accelerator Laboratory $\vert$ Spencer Gessner, Ph.D., Staff Scientist}
    \item Simulated plasma wakefield acceleration (PWFA) of positron beams using {HiPACE++} simulation code.
    \item Re-derived theoretical model of linear-regime PWFA and compared them to simulations analyzed in Python.
    \item Collaborated with physicists internationally ({SLAC}, {LBNL}, {DESY}) to tune simulation parameters.
    \item Read 15+ papers on plasma wakefield acceleration to understand the physics and current research.
    \item Wrote and submitted a manuscript to be \colhref{blue!65!black}{https://arxiv.org/abs/2311.07087}{published} in \textit{Journal of Physics: Conference Series}.
    \item \colhref{blue!65!black}{https://docs.google.com/presentation/d/1dLFNB72h5O4ACa0lj33pgTrrmjeLWT7p04UOhfNXdBw/edit?usp=sharing}{Presented} research to 30+ leading experts in plasma wakefield acceleration at FACET-II User Meeting.
\end{jobcustomlong}
\begin{jobcustomlong}{Computational Accelerator Physics Intern}{-- Liquid xenon positron source}{Jun 2022 -- Mar 2023}{{SLAC} National Accelerator Laboratory $\vert$ Spencer Gessner, Ph.D., Staff Scientist}
    \item Wrote code in C++ using GEANT4 library to simulate novel positron source. Analyzed results in Python.
    \item Developed engineering requirements for a liquid xenon positron source with physicists at {SLAC} and  Jefferson Lab. Consulted with physicist at {RIKEN} (Nishina) to conduct simulations.
    \item Wrote and \colhref{blue!65!black}{http://dx.doi.org/10.1016/j.nima.2023.168329}{published} peer-reviewed article in \textit{Nuclear Inst.\!\;and Methods in Physics Research Sec. A}.
    \item \colhref{blue!65!black}{https://docs.google.com/presentation/d/1vRQo0vVH0A9cEBvVVeSKDu2MXw4V9R3hdnxamKaW3T0/edit?usp=sharing}{Presented} research to 20+ accelerator physicists at Intl.\!\;Workshop on Future Linear Colliders. Presented \colhref{blue!65!black}{https://drive.google.com/file/d/1iXObaHtb2xi4kYt638eMDfPBv-0eD-CW/view?usp=sharing}{poster} at APS April Meeting.
\end{jobcustomlong}
\begin{jobcustomlong}{Soft Matter R\&D Research Assistant}{-- Memory physics}{Sep 2021 -- Jun 2022}{\calpoly\ $\vert$ Hilary Jacks, Ph.D., Assistant Professor}
    % \item Prototyped device to study hysteresis loops in knitted materials using Arduino.
    \item Designed and built linear actuator prototype with force sensor to measure hysteresis in knitted materials.
    \item Wrote custom Arduino (C++) code with documentation to control stepper motor and collect force data.
    % \item Soldered electrical components for custom stepper motor driver.
    % \item Wrote markdown documentation for future students to continue research.
\end{jobcustomlong}
\begin{jobcustomlong}{Computational Soft Matter Physics Research Assistant}{-- Memory physics}{Mar 2021 -- Aug 2021}{\calpoly\ $\vert$ Hilary Jacks, Ph.D., Assistant Professor}
    \item Studied information theory in 2D disordered systems using the Python library SwellPy.
    \item Implemented anisotropy functionalities in SwellPy to improve the memory capacity of disordered systems.
    % \item Modelled 2D disordered systems using Python to study information theory.
    % \item Added anisotropy functionalities to SwellPy Python library.
    \item \colhref{blue!65!black}{https://docs.google.com/presentation/d/1XdPL6FM-Nhcb5U-06zQodZRGSYDMoiufV3TQnuubFIQ/edit?usp=sharing}{Presented} at Frost Summer Physics Symposium; presented \colhref{blue!65!black}{https://docs.google.com/presentation/d/1gc6V4EIddg6weyhoOCH6Uc6ZFkzSfBEEbIQB2iCvSrg/edit?usp=sharing}{poster} at APS Far West Section Meeting. 
\end{jobcustomlong}
% \begin{jobcustomlong}{Numerical Quantum Mechanics Research Assistant}{}{Dec 2020 -- Jun 2021}{\calpoly\ $\vert$ Ben Shlaer, Ph.D., Lecturer}
%     \item Developed Mathematica code to numerically solve the time-independent Schr\"odinger equation.
%     \item Implemented the ``shooting method'' to numerically find energy eigenstates for user-drefined potentials.
% \end{jobcustomlong}
\vspace{-1.5em}
%----------------------------------------------------------------------------------------
%	PROGRAMMING
%----------------------------------------------------------------------------------------
% \section{Programming Skills}
% \begin{minipage}{0.7\linewidth}
% \begin{tabularx}{\linewidth}{@{}l X@{}}
% \textbf{Languages} & Python, C++, Bash/Zsh, JSON \\
% \textbf{Programs} & MATLAB, Mathematica, Microsoft Excel, SolidWorks \\
% \textbf{Markup} & \LaTeX, HTML, CSS, Markdown \\
% \textbf{Misc.} & Machine learning (Pytorch), Git, Grover's quantum search algorithm
% \end{tabularx}
% \end{minipage}
% \begin{minipage}{0.3\linewidth}
%     \hspace{5em}\textbf{Other Skills}
%     \begin{itemize}[nosep,after=\strut, leftmargin=6em, itemsep=3pt,label=--]
%         \item Soldering
%         \item Laser cutter
%         \item FDM 3D printing
%     \end{itemize}
% \end{minipage}
\section{Programming Skills}
\begin{tabularx}{\linewidth}{@{}l X@{}}
\textbf{Languages} & Python, C++, Bash/Zsh, JSON \\
\textbf{Programs} & MATLAB, Mathematica, Microsoft Excel, SolidWorks \\
\textbf{Markup} & \LaTeX, HTML, CSS, Markdown \\
\textbf{Misc.} & Machine learning (Pytorch, scikit-learn), High performance computation (LUMI, S3DF, NERSC), Git, PGF/TikZ
\end{tabularx}

%----------------------------------------------------------------------------------------
%	PROJECTS
%----------------------------------------------------------------------------------------
% \section{Projects}

% \begin{tabularx}{\linewidth}{ @{}l r@{} }
% \textbf{Some Project} & \hfill \href{https://some-link.com}{Link to Demo} \\[3.75pt]
% \multicolumn{2}{@{}X@{}}{long long line of blah blah that will wrap when the table fills the column width long long line of blah blah that will wrap when the table fills the column width long long line of blah blah that will wrap when the table fills the column width long long line of blah blah that will wrap when the table fills the column width}  \\
% \end{tabularx}

%----------------------------------------------------------------------------------------
%	PUBLICATIONS
%----------------------------------------------------------------------------------------
\section{Publications and Preprints}\label{sec:publications}
\begin{refsection}[citations.bib]
\nocite{*}
\printbibliography[heading=none]
\end{refsection}

%----------------------------------------------------------------------------------------
%	TALKS/PRESENTATIONS
%----------------------------------------------------------------------------------------
\section{Talks/Posters}\label{sec:talks}
\begin{tabularx}{\linewidth}{@{}l X@{}}
2025 & \colhref{blue!30!black}{https://indico.slac.stanford.edu/event/9594/}{{Updated simulations of a LXe $e^+$ target}} \hfill\textrm{\small US-Japan Advanced $e^+$ Source Concepts Meeting} \\
2024 & \colhref{blue!30!black}{https://github.com/MaxVarverakis/Representation-theory-in-physics/blob/main/Defense/Presentation.pdf}{{Representation Theory and its Applications in Physics}} \hfill\textrm{\small Cal Poly Master's Thesis Defense Colloquium} \\
2023 & \colhref{blue!30!black}{https://docs.google.com/presentation/d/1dLFNB72h5O4ACa0lj33pgTrrmjeLWT7p04UOhfNXdBw/edit?usp=sharing}{{Energy Recovery for Plasma-based Positron Acceleration}} \hfill\textrm{\small FACET-II User Meeting} \\
2023 & \colhref{blue!30!black}{https://docs.google.com/presentation/d/1vRQo0vVH0A9cEBvVVeSKDu2MXw4V9R3hdnxamKaW3T0/edit?usp=sharing}{{Liquid Xenon Positron Target}} \hfill\textrm{\small Intl. Workshop on Future Linear Colliders} \\
2023 & \colhref{blue!30!black}{https://meetings.aps.org/Meeting/APR23/Session/E01.19}{Liquid Xenon Positron Target} (\colhref{NavyBlue}{https://drive.google.com/file/d/1iXObaHtb2xi4kYt638eMDfPBv-0eD-CW/view?usp=sharing}{\textit{Poster}}) \hfill\textrm{\small APS April Meeting} \\
2021 & \colhref{blue!30!black}{https://meetings.aps.org/Meeting/FWS21/Session/N01.6}{Multiple Memories in an Anisotropic Swelling System} (\colhref{NavyBlue}{https://docs.google.com/presentation/d/1gc6V4EIddg6weyhoOCH6Uc6ZFkzSfBEEbIQB2iCvSrg/edit?usp=sharing}{\textit{Poster}}) \hfill\textrm{\small APS Far West Section} \\
2021 & \colhref{blue!30!black}{https://docs.google.com/presentation/d/1XdPL6FM-Nhcb5U-06zQodZRGSYDMoiufV3TQnuubFIQ/edit?usp=sharing}{Multiple Memories in an Anisotropic Swelling System} \hfill\textrm{\small Frost Summer Physics Symposium} \\
\end{tabularx}

%----------------------------------------------------------------------------------------
%	TEACHING
%----------------------------------------------------------------------------------------
\section{Teaching Experience}
% \begin{tabularx}{\linewidth}{l@{} X r@{}}
% {\textbf{Math 221} \textit{Calculus for Business and Economics}} & \hfill & \color{black!55!white}{Cal Poly, San Luis Obispo} \\
% {\small Spring 2023, Winter 2024, Spring 2024} \\

% \textbf{Responsibilities:} 
% \end{tabularx}
\begin{jobcustomlong}{Teaching Associate}{-- Calculus for Business and Economics}{Apr 2023 -- Jun 2024}{\calpoly}
    \item Taught 4-unit course MATH 221 for 3 quarters, totaling $\sim$370 hours and 160 students.
    \item Lectured 4 hours/class weekly. Wrote lesson plans, quizzes, and exams. Held 1 office hour/class weekly.
\end{jobcustomlong}
\vspace{-1.5em}

%----------------------------------------------------------------------------------------
%	AWARDS
%----------------------------------------------------------------------------------------
\section{Awards/Honors}
\begin{tabularx}{\linewidth}{@{}l X@{}}
    2024 & Received Master's Degree With Distinction \hfill {Cal Poly, Mathematics Dept.} \\
    2021 & Advancement of Science and Technology Scholarship \hfill {Cal Poly, Mathematics Dept.} \\
    2021 & Top Applicant Stipend \hfill {CMAP Summer School Program} \\
\end{tabularx}


\vfill
\center{\footnotesize Last updated: \today}

\end{document}
